\documentclass[UTF8]{ctexart}
\usepackage{mathrsfs}
\usepackage{amsfonts}
\usepackage{graphicx}
\newtheorem{thm}{定理}
\newcommand\degree{^\cric}
\begin{document}
\title{杂谈勾股定理}
\author{刘凯}
\date{\today}
\maketitle
\begin{abstract}
\small\centering{这是有关勾股定理的小论文}
\end{abstract}
\tableofcontents
\section{勾股定理在古代}
\subsection{古代}
\label{sec:ancient}
\small 在中国,商朝时期的商高提出了“勾三股四玄五”的勾股定理的特例。在西方,最早提出并证明此定理的为公元前6世纪古希腊的毕达哥拉斯学派,他用演绎法证明了直角三角形斜边平方等于两直角边平方之和
\subsection{近代}
\small 远在公元前约三千年的古巴比伦人就知道和应用勾股定理,他们还知道许多勾股数组。美国哥伦比亚大学图书馆内收藏着一块编号为“普林顿322”的古巴比伦泥板,上面就记载了很多勾股数。古埃及人在建筑宏伟的金字塔和测量尼罗河泛滥后的土地时,也应用过勾股定理 



\footnotesize\centering 勾三股四玄

\small 又载陈子(公元前7——6世纪)

\section{勾股定理的近代形式} 


%\begin{thm}[\small勾股定理]
%    \small 直角边斜边的平方等于两腰的平方和。
%    \small 可以用符号语言表示为:设直角三角形$ABC$,其中$\angle C=90degree$ ,则有
%    \begin{equation}\label{eq:gougu}
%    \small AB^2=BC^2+AC^2.
%        \end{thm}[\small勾股定理]
%        \small 满足式 \eqref{eq:gougu}  的整数称为\emph{勾股数}。第\ref{sec:ancient} 节所说的流派得到的即使勾股数 
%        
%        vspace{3mm}
%        \begin{tabular}{|c|c|c|}\hline
%        \small 直角边$a$&直角边$b&$斜边$c$
%        3&4&5\\\hline
%        5&12&13\\\hline
%        \end{tabular}
%        \small($a^2+b^2=c^2$)
%
\addcontentsline{toc}{section}{参考文献}
\begin{thebibliography}{99}
   
\bibitem{1}矢野健太郎.集合的有名定理.上海某出版社,1986.
\bibitem{quanjing}曲安金.商高、赵爽和刘辉关于勾股定理的证明.数学传播,20(3),1998.
\end{thebibliography}

\begin{appendix}
    \section{附录}·
\small 勾股定理又叫商高定理,国外也称百牛定理
\end{appendix}
\end{document}
